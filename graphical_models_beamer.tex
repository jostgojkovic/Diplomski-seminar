\documentclass{beamer}
\usepackage[slovene]{babel}
\usepackage[utf8]{inputenc}
\usepackage[T1]{fontenc}
\usepackage{array}              %brez tega paketa ne bo delala točka: odkrivanje po stolpcih
\usepackage{palatino}
\usepackage{graphicx}
\usepackage{amsmath}
\usepackage{amsthm}
\usepackage{tikz}

\usepackage{relsize}

\beamertemplatenavigationsymbolsempty     %s tem ukazom skriješ gumbe za navigacijo


\usetheme{Berlin}
\usefonttheme{serif}
\usecolortheme{default}
\useinnertheme[shadows]{rounded}
\useoutertheme{infolines}

\newtheorem{definicija}{Definicija}
\newtheorem{izrek}{Izrek}




\title{Diplomski seminar}
   \subtitle{Modeliranje pogojne neodvisnosti s pomočjo grafov}
   \author{Jošt Gojkovič}
   \institute[FMF] {Fakulteta za matematiko in fiziko}
   \date{12. 12. 2022}                                               

\begin{document}




% ===================================================================
\begin{frame}
    \titlepage         %če tega ni, se un \title ne pokaže = naslovnica
  \end{frame}
  

% -------------------------------------------------------------------
\section{Motivacija}

\begin{frame}
    \frametitle{Grafično modeliranje}
    Grafično modeliranje se uporablja v:
    \begin{itemize}
        \item statistični fiziki 
        \item genetiki 
        \item še kej 
    \end{itemize}
   
\end{frame}

% -------------------------------------------------------------------
\section{Teorija Grafov}

\begin{frame}
    \frametitle{Notacija in terminologija}
    $G(V,E)$ enostaven graf, torej nima večkratnih povezav in zank. Večinoma bodo 
    vozlišča označena, to je bodo razdeljena v 2 skupini.
    \begin{itemize}
        \item Množica vozlišč ima obliko 
        $$ V = \Delta \cup \Gamma \qquad \text{z} \qquad \Delta \cap  \Gamma = \emptyset  $$ 
        Pravimo, da so vozlišča v $\Delta$ diskretna, v $\Gamma$ pa zvezna.  
        \item Grafi z označenimi vozlišči so označeni grafi 
    \end{itemize}
\end{frame}
% -------------------------------------------------------------------

\begin{frame}
    \frametitle{Notacija in terminologija}
    \begin{itemize}
        \item Poln graf je graf, v katerem vsaka povezava povezuje par njegovih točk,
        oziroma kjer so vse točke povezane vsaka z vsako.
        \item  Če je $A \subseteq V$, $A$ inducira podgraf $G_A = (A, E_A)$,
        kjer je $E_A = E \cap (A \times A)$ dobljen iz $G$ tako da ohranimo povezave z 
        začetnim in končnim vozliščem v $A$.
        \item Podmnožica je polna, če inducira poln podgraf
        \item Polni podmnožici, ki je maksimalna oziroma se je ne da povečat, pravimo klika
   
    \end{itemize}

\end{frame}
% -------------------------------------------------------------------
\begin{frame}
    \frametitle{Notacija in terminologija}
    \begin{itemize}
        \item Če imamo povezavo $ \alpha \longrightarrow \beta $, pravimo, da je $\alpha$ starš od $\beta$.
        Množica staršev vozlišča $\beta$ je označena s $pa(\beta)$.  
        \item Množico sosedov vozlišča $\alpha$ označimo z $ne(\alpha)$
        \item Oznaki $pa(A)$ in $ne(A)$ pa označujeta množico staršev in sosedov vozlišč v $A$,
        katera sama niso v $A$:
        \begin{align*}
              pa(A) = \cup_{\alpha \in A} pa(\alpha) \setminus A \\
              ne(A) = \cup_{\alpha \in A} pa(\alpha) \setminus A
        \end{align*}
        \item \emph{Meja} $bd(A)$ podmnožice vozlišč $A$ je množica vozlišč v $ V \setminus A $,
        ki so starši ali sosedi vozliščem v $A$. Torej $ bd(A) = pa(A) \cup ne(A)$. 
        \item \emph{Zaprtje} množice $A$ je $cl(A) = A \cup bd(A) $
    \end{itemize}

\end{frame}
% -------------------------------------------------------------------
\begin{frame}
    \frametitle{Notacija in terminologija}
    \begin{itemize}
        \item Množica $C \subseteq V $ je $ (\alpha, \beta)$-seperator, če vse poti od $\alpha$ do $\beta$ sekajo 
        množico $C$.
        \item $C \subseteq V $ separira $A$ od $B$, če je $ (\alpha, \beta)$-seperator za vse $\alpha \in A$
        in $\beta \in B$
    \end{itemize}
    \begin{definicija}
        Trojica $(A, B, C)$ disjunktnih podmnožic vozlišč neusmerjenega označenega garfa $G$ tvori
        močno dekompozicijo grafa $G$, če je $V = A \cup B \cup C$ in velja naslednje:
        \begin{enumerate}[(i)]
            \item $C$ separira $A$ od $B$
            \item $C$ je polna podmnožica množice $V$
            \item $C \subseteq \Delta \lor B \subseteq \Gamma$
        \end{enumerate}
    \end{definicija}
    \begin{itemize}
        \item Če pri enakih predpostavkah veljata samo 1. in 2. pogoj, potem trojica  $(A, B, C)$
        tvori šibko dekompozicijo.
        \item Pravimo, da $(A, B, C)$ dekompozira $G$ v komponenti $G_{A \cup C}$ 
        in $G_{B \cup C} $
    \end{itemize}

\end{frame}
% -------------------------------------------------------------------

% -------------------------------------------------------------------


\end{document}